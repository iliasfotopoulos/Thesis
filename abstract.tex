\begin{abstractgr}
	Τα τελευταία χρόνια έχει παρατηρηθεί μια ραγδαία αύξηση στις ανάγκες για αίμα και για τα προϊόντα του αίματος γενικότερα, η οποία αναμένεται να συνεχίσει να αυξάνεται στα επόμενα χρόνια. Για να καλυφθούν οι όλο και μεγαλύτερες ανάγκες είναι μονόδρομος η κινητοποίηση, η ευαισθητοποίηση και η καλύτερη οργάνωση των εθελοντών έτσι ώστε να επιτευχθεί μεγαλύτερος αριθμός εθελοντικών αιμοδοσιών και κατ' επέκταση αυτάρκεια. Η παρούσα διπλωματική αφορά τον σχεδιασμό και υλοποίηση ενός ολοκληρωμένου συστήματος αιμοδοσίας\symbol{"0387} το οποίο έχει ως στόχο να βοηθήσει σημαντικά στην εξεύρεση νέων εθελοντών, αλλά και ταυτόχρονα να διατηρήσει τους υπάρχοντες αυξάνοντας την συστηματικότητα τους και τον αριθμό των δωρεών που πραγματοποιούν. Μεγάλη βαρύτητα δόθηκε στον σχεδιασμό της αρχιτεκτονικής του συστήματος και διασύνδεσης του με τα διάφορα υποσύστημα τα του, αξιοποιώντας τις κρατούσες
τεχνολογικές προσεγγίσεις περί ανάπτυξης συστημάτων πολλαπλών επιπέδων. Επίσης έχει γίνει εκτενής μελέτη και χρήση της τεχνολογίας υπολογιστικού νέφους λόγω της μεγάλης ελαστικότητας, της επεκτασιμότητας και της οικονομίας κλίμακας που προσφέρει.  Συγκεκριμένα το σύστημα αποτελείται από τα παρακάτω διασυνδεδεμένα υποσυστήματα: i) Εφαρμογή έξυπνου κινητού τηλεφώνου (smartphone) προοριζόμενη για χρήση από τους εθελοντές αιμοδότες ii) Διαδικτυακή εφαρμογή (cloud portal) για διαχείριση των αιτημάτων αιμοδοσίας από τα κέντρα αιμοδοσίας  iιι) Υποσύστημα ελέγχου του ιατρικού ιστορικού  του υποψήφιου εθελοντή αιμοδότη μέσω ανταλλαγής δεδομένων με χρήση πρωτοκόλλων HL7 και CDA Documents και απόφανσης της καταλληλότητας του ή μη. Για το υποσύστημα ανταλλαγής δεδομένων μελετήθηκαν τα ζητήματα διαλειτουργικότητας με τρίτα συστήματα και προτάθηκαν μηχανισμοί διαλειτουργικότητας που εξυπηρετούν τις ανάγκες ανταλλαγής δεδομένων μεταξύ συστημάτων.
    \begin{keywordsgr}
		εθελοντική αιμοδοσία, πληροφοριακό σύστημα αιμοδοσίας, μητρώο αιμοδοτών, αιμοεπαγρύπνηση, εφαρμογή κινητού, cloud computing, clinical decision support system
	\end{keywordsgr}
\end{abstractgr}

\begin{abstracten}
	\begin{keywordsen}
	\end{keywordsen}
\end{abstracten}

\begin{acknowledgementsgr}

\end{acknowledgementsgr}
