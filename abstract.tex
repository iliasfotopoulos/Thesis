\clearpage

\thispagestyle{empty}
\null\vfill

\settowidth\longest{\huge\itshape Nulla dies sine linea}
\begin{center}
	\parbox{\longest}{%
	  \raggedright{\huge\itshape%
	   Nulla dies\\ sine linea.\par\bigskip
	  }   
	  \raggedleft\Large\MakeUppercase{Apelles}\par%
	}
\end{center}
\vfill\vfill

\clearpage


\begin{acknowledgementsgr}
Με τη λήξη αυτού του ακαδημαϊκού κύκλου, θέλουμε να ευχαριστήσουμε όλους όσοι στάθηκαν στο πλευρό μας και μας στήριξαν κατά τη διάρκεια της εκπόνησης της διπλωματικής μας εργασίας αλλά και καθ’ όλη τη διάρκεια των σπουδών μας .

Θα θέλαμε αρχικά να ευχαριστήσουμε τον επιβλέποντα καθηγητή, κ. Δημήτριο Κουτσούρη, που μας έδωσε τη δυνατότητα να ασχοληθούμε με ένα τόσο ενδιαφέρον και σημαντικό θέμα.

Επίσης θα θέλαμε να ευχαριστήσουμε τον κ. Ιωάννη Κουρή για τη συνεχή και αποτελεσματική βοήθειά του, καθώς και τη συνολική υποστήριξή του καθ΄ όλη τη διάρκεια εκπόνησης της διπλωματικής μας εργασίας.

Τέλος, οφείλουμε ένα μεγάλο ευχαριστώ στις οικογένειές μας, στους γονείς μας Βασίλη-Κέλλυ και Κωνσταντίνο-Βασιλική αντίστοιχα, καθώς και στους φίλους μας, για την αμέριστη συμπαράσταση, υποστήριξή και συνολική τους βοήθεια σε όλα τα επίπεδα.
\end{acknowledgementsgr}

\begin{abstractgr}
	Τα τελευταία χρόνια έχει παρατηρηθεί μια ραγδαία αύξηση στις ανάγκες για αίμα και στα προϊόντα του αίματος γενικότερα, η οποία αναμένεται να συνεχίσει να κλιμακώνεται τα επόμενα χρόνια. Για να καλυφθούν οι όλο και μεγαλύτερες ανάγκες είναι μονόδρομος η κινητοποίηση, η ευαισθητοποίηση και η καλύτερη οργάνωση των εθελοντών έτσι ώστε να επιτευχθεί μεγαλύτερος αριθμός εθελοντικών αιμοδοσιών και κατ' επέκταση αυτάρκεια. Η παρούσα διπλωματική αφορά στον σχεδιασμό και στην υλοποίηση ενός ολοκληρωμένου συστήματος αιμοδοσίας\symbol{"0387} το οποίο έχει ως στόχο να βοηθήσει σημαντικά στην εξεύρεση νέων εθελοντών, αλλά και ταυτόχρονα να διατηρήσει τους υπάρχοντες, αυξάνοντας την συστηματικότητα τους και τον αριθμό των δωρεών που πραγματοποιούν. Μεγάλη βαρύτητα δόθηκε στον σχεδιασμό της αρχιτεκτονικής του συστήματος και στη διασύνδεση του με τα διάφορα υποσυστήματα με τα οποία αλληλεπιδρά, αξιοποιώντας τις κρατούσες τεχνολογικές προσεγγίσεις περί ανάπτυξης συστημάτων πολλαπλών επιπέδων. Επίσης έχει γίνει εκτενής μελέτη και χρήση της τεχνολογίας υπολογιστικού νέφους λόγω της μεγάλης ελαστικότητας, της επεκτασιμότητας και της οικονομίας κλίμακας που προσφέρει.  Συγκεκριμένα το σύστημα αποτελείται από τα παρακάτω διασυνδεδεμένα υποσυστήματα: i) Εφαρμογή έξυπνου κινητού τηλεφώνου (smartphone) προοριζόμενη για χρήση από τους εθελοντές αιμοδότες ii) Διαδικτυακή εφαρμογή (cloud portal) για διαχείριση των αιτημάτων αιμοδοσίας από τα κέντρα αιμοδοσίας  iii) Υποσύστημα ελέγχου του ιατρικού ιστορικού  του υποψήφιου εθελοντή αιμοδότη μέσω ανταλλαγής δεδομένων με χρήση πρωτοκόλλων HL7 και CDA Documents και απόφανσης της καταλληλότητας του ή μη. Για το υποσύστημα ανταλλαγής δεδομένων μελετήθηκαν τα ζητήματα διαλειτουργικότητας με τρίτα συστήματα και προτάθηκαν μηχανισμοί διαλειτουργικότητας που εξυπηρετούν τις ανάγκες ανταλλαγής δεδομένων μεταξύ συστημάτων.
    \begin{keywordsgr}
		εθελοντική αιμοδοσία, πληροφοριακό σύστημα αιμοδοσίας, μητρώο αιμοδοτών, αιμοεπαγρύπνηση, εφαρμογή κινητού,υπολογιστικό νέφος, συστήματα υποστήριξης κλινικών αποφάσεων
	\end{keywordsgr}
\end{abstractgr}

\begin{abstracten}
	Over the last few years we have witnessed a surge in demand of blood and generally of blood products. We expect the paucity of blood products to be further exacerbated by the ageing of the population. Efforts should be made in order to establish and maintain sufficient numbers of regular, volunteer blood donors to ensure an adequate and safe blood supply. In this thesis we aim to synergistically exploit mobile and cloud computing in order to design and implement a cloud-based blood donation management information system. This system has as primary goal to actively help in the recruitment and retention of blood donors by taking advantage of mobile technologies and gamification. Emphasis was given in the architectural design for the system by combining current technological architectures based on a multi-tier architecture and cloud-based software development. In this context we have extensively studied cloud computing and the advantages that can offer to the health sector. Particulary the system that has been implemented as part of this thesis consists of the following interconnected subsystems: i) Smartphone application  ii) Cloud application iii) Clinical decision support subsystem that helps to determine the eligibility of the blood donor based on his medical history. In the context of the development of the aforementioned subsystem, we have also studied issues related to interoperability with third party systems and we proposed interoperability mechanisms based on international standards and best practices.
	\begin{keywordsen}
		blood donation, voluntary blood donation, blood donation management information system, haemovigilance, cloud computing, clinical decision support system
	\end{keywordsen}
\end{abstracten}

