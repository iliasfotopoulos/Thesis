\graphicspath{ {Figures/technology_stack/} }
\chapter{Τεχνολογίες}\label{ch:Development Stack}
\section{Back-End}
	\subsection{Cloud}
	

	To cloud computing (υπολογιστικό νέφος) όπως ορίζεται από το NIST (Εθνικό Ινστιτούτο Προτύπων και Τεχνολογίας)  είναι μια τεχνολογία, η οποία επιτρέπει την πρόσβαση από παντού, παρέχει μετά από αίτημα πρόσβαση στο δίκτυο για την κοινή χρήση προσαρμοστικών υπολογιστικών πόρων (π.χ., δίκτυα, servers, συστήματα αποθήκευσης, εφαρμογές και υπηρεσίες), μπορεί να ξεκινήσει και να αναπτυχθεί γρήγορα με ελάχιστη διαχείριση και χωρίς καμία αλληλεπίδραση με τον φορέα παροχής υπηρεσιών. \cite{cloudComputing} Το cloud computing παρέχει στους χρήστες και στις επιχειρήσεις διάφορες δυνατότητες για την αποθήκευση και την επεξεργασία των δεδομένων τους σε τρίτα κέντρα δεδομένων. \cite{Haghighat2015}
	Το cloud computing ουσιαστικά στηρίζεται στην κατανομή των πόρων με στόχο την επίτευξη συνοχής και οικονομίας μέσω ενός δικτύου και την μεγιστοποίηση της απόδοσης τους. Οι πόροι του cloud δεν είναι μόνο διαμοιραζόμενοι από πολλούς χρήστες αλλά μπορούν και να ανακατανεμηθούν δυναμικά ανάλογα με την ζήτηση, γεγονός που μπορεί να χρησιμοποιηθεί όταν κατανέμονται στους διάφορους χρήστες.  Για παράδειγμα, μια cloud εγκατάσταση ηλεκτρονικών υπολογιστών οι οποία εξυπηρετεί Ευρωπαίους χρήστες στην διάρκεια των Ευρωπαϊκών εργάσιμων ωρών μπορεί να ανακατανείμει τους ίδιους πόρους για να εξυπηρετήσει χρήστες στην Βόρεια Αμερική κατά τις εργάσιμες ώρες της Βόρειας Αμερικής σε μία διαφορετική εφαρμογή. Αυτή η προσέγγιση βοηθά στην μεγιστοποίηση της χρήσης της υπολογιστικής ισχύος ενώ ταυτόχρονα μειώνεται το ολικό κόστος των φυσικών πόρων καθώς χρησιμοποιούμε λιγότερη ενέργεια, κλιματισμό κ.λ.π. για να διατηρείται το σύστημα.Με χρήση του cloud computing πολλοί χρήστες μπορούν να έχουν πρόσβαση σε έναν server και να ανακτούν και να ανανεώνουν τα δεδομένα τους χωρίς να αγοράζουν άδειες για όλες τις διαφορετικές εφαρμογές. 
	Οι πάροχοι cloud συνήθως χρησιμοποιούν μοντέλο κοστολόγησης "πλήρωσε όσο χρησιμοποιήσεις".  Παρουσιάζει σπουδαιότητα λοιπόν το γεγονός ότι οι "πελάτες" μπορούν να εξοικονομήσουν χρηματικούς πόρους στο πέρασμα του χρόνου, καθώς χρησιμοποιώντας τους πόρους στο cloud, πληρώνουν μόνο για τους πόρους που χρησιμοποιούν και όχι για την αγορά και συντήρηση άφθονων μηχανημάτων (π.χ. εξυπηρετητών). Η λύση λοιπόν του cloud computing λοιπόν αποτελεί μία ιδιαίτερα ελκυστική προοπτική για τους χρήστες στους οποίους η χρήση των πόρων ποικίλλει σημαντικά καθώς και σε αυτούς στους οποίους  η αγορά μηχανημάτων και το κόστος συντήρησης τους αποτελούν σημαντικό τμήμα του προϋπολογισμού. 

	Ένα πλήρες σύστημα υπολογιστικού νέφους αποτελείται από τα εξής βασικά συστατικά : 
	\begin{itemize}
	\item Πελάτες - υπολογιστές. Πρόκειται για τις συσκευές των τελικών χρηστών μέσω των οποίων αποκτούν πρόσβαση στο υπολογιστικό νέφος. Χωρίζονται σε τρεις κατηγορίες: τις κινητές συσκευές (mobile devices) που επιτρέπουν στον χρήστη την απομακρυσμένη πρόσβαση μετά από επεξεργασία του κατανεμημένου διακομιστή, τους παχείς πελάτες (thick clients) που είναι οι σταθεροί υπολογιστές ή τα λάπτοπ τα οποία συνδέονται στο cloud με χρήση ενός φυλλομετρητή και τους λεπτούς πελάτες (thin clients ) που είναι υπολογιστές με μικρή υπολογιστική ισχύς και χωρίς σκληρούς δίσκους, που εμφανίζουν τα δεδομένα αφού τα επεξεργαστεί πρώτα ο διακομιστής.
	\item Κέντρο δεδομένων. Είναι το σύνολο των διακομιστών, όπου τρέχουν οι εφαρμογές. Στο κέντρο δεδομένων εμπίπτουν και οι εικονικοί διακομιστές (μπορεί σε έναν υπολογιστή να γίνει εγκατάσταση ώστε να επιτρέπεται η ταυτόχρονη ύπαρξη πολλαπλών στιγμιοτύπων από εικονικούς διακομιστές).
	\item Κατανεμημένους διακομιστές. Οι διακομιστές αυτοί βρίσκονται σε διαφορετικές γεωγραφικές θέσεις, έτσι ώστε να εξασφαλίζεται μεγαλύτερη ευελιξία και υψηλότερη ασφάλεια στις παροχές που προσφέρει το cloud computing. Έτσι μπορούν να αντιμετωπιστούν αποτελεσματικά περιπτώσεις παραβίασης ασφάλειας ή ανάγκης επιπλέον αποθηκευτικού χώρου και υπολογιστικής δύναμης.
	\end{itemize}		
	\
		Το μοντέλο cloud αποτελείται από πέντε βασικά χαρακτηριστικά:
		\begin{itemize}
		\item Αυτοεξυπηρέτηση μετά από ζήτηση (On-demand self-service). Ο χρήστης των υπηρεσιών του νέφους μπορεί να χρησιμοποιεί τους υπολογιστικούς πόρους που χρειάζεται, όπως χρόνο στον server και αποθηκευτικό χώρο,  αυτόματα και χωρίς να απαιτείται ανθρώπινη αλληλεπίδραση με τον φορέα παροχής υπηρεσιών. 
		\item Ευρεία πρόσβαση στο δίκτυο (Broad network access). Οι δυνατότητες που προσφέρονται είναι διαθέσιμες μέσω του δικτύου και προσβάσιμες μέσω τυποποιημένων μηχανισμών που προωθούν την χρήση ετερογενών πλατφορμών λεπτών ή παχέων πελατών (π.χ., κινητά τηλέφωνα, τάμπλετ, υπολογιστές). 
		\item Διάθεση των πόρων. Οι υπολογιστικοί πόροι των cloud παρόχων είναι συγκεντρωμένοι με χρήση ενός μοντέλου πολλαπλών ενοικιαστών (κάθε οργανισμός εργάζεται με ένα προσαρμοσμένο εικονικό στιγμιότυπο της εφαρμογής) έτσι ώστε να  μπορούν να ανατεθούν οι φυσικοί και εικονικοί πόροι ανάλογα με την ζήτηση των καταναλωτών. Οι παρεχόμενοι πόροι μπορεί να περιλαμβάνουν πόρους όπως μνήμη, επεξεργαστική ισχύ, εύρος δικτύου, εικονικές μηχανές.
		\item Μεγάλη ελαστικότητα (Rapid Elasticity). Οι δυνατότητες του cloud computing παρέχονται με μεγάλη ταχύτητα και προσαρμοστικότητα , ώστε να είναι σε θέση να κλιμακώνουν πολύ γρήγορα ανάλογα με την ζήτηση που υπάρχει και να φαίνονται απεριόριστες σε οποιαδήποτε ποσότητα και χρονική στιγμή ζητηθούν. 
		\item Μετρούμενη υπηρεσία (Measured Service). Τα cloud computing συστήματα πραγματοποιούν μετρήσεις, σε συγκεκριμένα επίπεδα αφαίρεσης ανάλογα με το είδος της υπηρεσίας (επεξεργασία, μνήμη, ενεργοί λογαριασμοί χρηστών). Με βάση αυτές τις μετρήσεις γίνεται έλεγχος και ενέργειες για την βελτιστοποίηση της χρήσης των πόρων. Επιπλέον η χρήση των πόρων παρακολουθείται, ελέγχεται και καταγράφεται με αποτέλεσμα την παροχή διαφάνειας μεταξύ του χρήστη και του παρόχου του cloud.
		\end{itemize}	\cite{characteristicsCloud}\cite{nist}.

	
		\subsubsection{IaaS}
		\subsubsection{PaaS}
		\subsubsection{SaaS}
		\subsubsection{BaaS}
			Γιατί Parse ?
	Μοντέλα ανάπτυξης
		\subsubsection{Private cloud}
		\subsubsection{Community cloud}
		\subsubsection{Public cloud}
		\subsubsection{Hybrid cloud}
		\subsubsection{NodeJs}
		\subsubsection{Cloud και ασφάλεια Ιατρικών δεδομένων?}
	\subsection{Βάση Δεδομένων}
		\subsubsection{Σχετικά με NoSql}
			NoSql vs Relation DB - βρες κάτι και για medical με nosql. Γιατί NoSql. Γιατί MongoDb.
		\subsubsection{Schema}

\section{Front-End}
	\subsection{Αρχιτεκτονική}
		\subsubsection{MVC αρχιτεκτονικό πρότυπο}\label{sssection:mvc}

		Το σχεδιαστικό μοτίβο Model View Controller (MVC) είναι ένα αρχιτεκτονικό πρότυπο που χρησιμοποιείται στην τεχνολογία λογισμικού και συνήθως στην υλοποίηση web εφαρμογών. Χωρίζει την εφαρμογή σε τρία διασυνδεδεμένα μέρη, με σκοπό τον πλήρη διαχωρισμό της εσωτερικής αναπαράστασης  της πληροφορίας στο σύστημα από τους τρόπους παρουσίασης της πληροφορίας στον χρήστη. Το MVC αποτελείται από τρία στοιχεία: α) Model, β) View, γ) Controller.
		\begin{itemize} 
		\item\textbf{Model:} Το model είναι το κομμάτι της εφαρμογής που αναπαριστά την πληροφορία, δηλαδή τα δεδομένα που χρησιμοποιεί η εφαρμογή και επιβάλλει τις ενέργειες και τους περιορισμούς πάνω σε αυτά.  Στο model τοποθετούμε τις λειτουργίες της εφαρμογής που σχετίζονται με την πρόσβαση στη βάση δεδομένων. Οι λειτουργίες αυτές είναι συναρτήσεις με τις οποίες διαχειριζόμαστε αποτελεσματικά τα δεδομένα που λαμβάνουμε από τη βάση.  Το μοντέλο ανταποκρίνεται σε ερωτήματα από το view καθώς και σε οδηγίες από τον controller για να πραγματοποιήσει ενημερώσεις.
		\item\textbf{View:} Αντιστοιχεί στην γραφική παρουσίαση των δεδομένων του model στο interface του χρήστη. Ο controller, είναι αυτός ο οποίος αποφασίζει πότε θα παρουσιαστούν τα δεδομένα ενώ το view καθορίζει τον τρόπο με τον οποίο θα παρουσιαστούν τα δεδομένα στον χρήστη. Το επίπεδο view θα πρέπει να προσαρμόζεται στις αλλαγές του επιπέδου του model, για αυτό και τα views θα πρέπει να επικεντρώνονται μόνο στην εμφάνιση δεδομένων και να μην εμπλέκονται με την επιχειρηματική λογική του model.
		\item\textbf{Controller:} Ο controller είναι αυτός που παίρνει αποφάσεις, δηλαδή διαχειρίζεται τη σύνδεση των δεδομένων του model με τη λογική του προγράμματος και καθορίζει ποια από τα δεδομένα του model θα παρουσιαστούν από το view στο interface του χρήστη. Ο controller ενημερώνει το view κάθε φορά που αλλάζει το model. Επιπλέον προσθέτει event listeners στο view και ενημερώνει το model όταν ο χρήστης αλληλεπιδρά με το view.
	 \end{itemize}		
	 \begin{figure}[h]
	    \centering
	    \includegraphics[width=0.7\textwidth]{mvcpattern.png}
	    \caption{Αρχιτεκτονική MVC.}
	    \label{fig:mvc_pattern}
	\end{figure}
		
		
		Η  χρήση της αρχιτεκτονικής MVC κατά την δημιουργία μίας εφαρμογής μας προσφέρει τα εξής βασικά πλεονεκτήματα:
		\begin{itemize}
		\item Διαχωρισμός των προβλημάτων (Separation of Concerns). Ουσιαστικά δημιουργείται μία εφαρμογή η οποία έχει τρία επίπεδα(models, views, controllers) και το κάθε επίπεδο επιτελεί ξεχωριστό έργο και ταυτόχρονα συνεργάζεται με τα άλλα επίπεδα. Στην σωστή υλοποίηση πρέπει τα τρία επίπεδα να είναι πλήρως καθορισμένα και να μην συμπλέκονται. Ο διαχωρισμός αυτός, επιτρέπει την επαναχρησιμοποίηση της επιχειρηματικής λογικής σε διάφορες εφαρμογές και την ανάπτυξη πολλαπλών user interfaces χωρίς να προβληματιζόμαστε από τον κώδικα που διαχειρίζεται την βάση.
		\item \textbf{Επεκτασιμότητα (Scaleability)}. Είναι η δυνατότητα που διαθέτει μία εφαρμογή , να μπορούμε μελλοντικά να προσθέσουμε λειτουργίες σε αυτή ή να αλλάξουμε κάποιες από τις ήδη υπάρχουσες λειτουργίες με σκοπό να επιτύχουμε διαφορετικά αποτελέσματα. 
		\item \textbf{Ελεγξιμότητα (Testability)}. Οι MVC εφαρμογές έχουν την δυνατότητα να είναι πιο εύκολα ελέγξιμες και με τον τρόπο αυτό συντηρούνται πιο εύκολα.  Εφόσον τα συστατικά της αρχιτεκτονικής είναι διακριτά είναι πιο εύκολο να γράφουμε κώδικα που να τεστάρει ξεχωριστά το κάθε κομμάτι, γρήγορα και αποτελεσματικά.
		\item \textbf{Παράλληλη ανάπτυξη από διαφορετικές ομάδες}. Οι προγραμματιστές επιχειρηματικής λογικής  μπορούν να δημιουργούν τις κλάσεις, ενώ οι προγραμματιστές του user interface μπορούν να σχεδιάζουν ταυτόχρονα τις οθόνες. Επίσης μπορούν να γίνονται ενημερώσεις και αλλαγές στο user interface χωρίς να επιβραδύνεται η διαδικασία του business logic.
		\end{itemize}
		
	\subsubsection{MVP αρχιτεκτονικό πρότυπο}
	Το αρχιτεκτονικό πρότυπο Model View Presenter (MVP) αποτελεί ένα παράγωγο αρχιτεκτονικό πρότυπο του MVC \cite{mvpPotel}, το οποίο είναι πολύ δημοφιλές στην ανάπτυξη εφαρμογών για το λειτουργικό σύστημα Android. Στην θεωρία τουλάχιστον, οι εφαρμογές Android είναι στημένες με την λογική του MVC \cite{androidArchAnalysis} όπως περιγράφηκε στην ενότητα \ref{sssection:mvc}, με το Activity ή το Fragment να αποτελεί τον controller και τα xml layouts τα views. Στην πράξη όμως καταλήγει η πλειονότητα του view logic να βρίσκεται στο Activity το οποίο το μετατρέπει σε δομή model-view όπως φαίνεται στο σχήμα \ref{fig:model_view}, κάτι το οποίο σε καμία περίπτωση δεν είναι το επιθυμητό. Στόχος λοιπόν του MVP είναι να λύσει ακριβώς αυτό το πρόβλημα και να πετύχει πλήρη διαχωρισμό μεταξύ του view και του view logic όπως βλέπουμε στο σχήμα \ref{fig:mvp_pattern}.
		
	 \begin{figure}[h]
	    \centering
	    \includegraphics[width=0.7\textwidth]{model_view.png}
	    \caption{Android Model-View αρχιτεκτονικό πρότυπο}
	    \label{fig:model_view}
	\end{figure}
	
	\begin{figure}[h]
	    \centering
	    \includegraphics[width=0.7\textwidth]{mvp_pattern.png}
	    \caption{Android MVP αρχιτεκτονικό πρότυπο}
	    \label{fig:mvp_pattern}
	\end{figure}
	
	Πλεονεκτήματα και λόγοι χρήσης του αρχιτεκτονικού προτύπου Model View Presenter\cite{androidHacks}:
	\begin{itemize}
		\item \textbf{Επεκτασιμότητα} η οποία επιτυγχάνεται μέσω του πλήρη διαχωρισμού μεταξύ της εμφάνισης (view) και της λογικής της εμφάνισης (view logic). 
		\item \textbf{Ευκολία Testing} του κάθε στρώματος (layer) της εφαρμογής αφού έχουμε πετύχει σαφή διαχωρισμό μεταξύ τους. Εν γένει η χρήση του μοτίβου MVP προάγει το test driven development της εφαρμογής με όλα τα πλεονεκτήματα που συνεπάγεται η εν λόγω τεχνική ανάπτυξης.
	\end{itemize}
	
	Το αρχιτεκτονικό πρότυπο MVP όπως δηλώνει και το όνομα του αποτελείται από τα παρακάτω τρία στοιχεία α) Model β) View γ) Presenter :
	\begin{itemize}
		\item \textbf{Model: } Το model στο αρχιτεκτονικό πρότυπο MVP επιτελεί της ίδιες λειτουργίες με αυτές στο MVC όπως περιγράφηκαν παραπάνω στην ενότητα \ref{sssection:mvc}.
		\item \textbf{View: } Το view υλοποιείται συνήθως μέσω κάποιου Activity ή Fragment (εξαρτάται από την συγκεκριμένη αρχιτεκτονική της εφαρμογής), και περιέχει αναφορά στον presenter ο οποίος το χειρίζεται. H εν λόγω αναφορά πραγματοποιείται με χρήση του σχεδιαστικού μοτίβου dependency injection και ιδανικά με χρήση του Dagger. Κάθε φορά που ο χρήστης αλληλεπιδρά με κάποια λειτουργία (π.χ κουμπί) του view, καλείται η αντίστοιχη συνάρτηση του presenter ο οποίος επικοινωνεί με το model και λέει εν τέλει στο view τι θα εμφανίσει.
		\item \textbf{Presenter: } Ο Presenter είναι λειτουργεί ως το ενδιάμεσο στάδιο - ενορχηστρωτής μεταξύ του view και του model. Είναι υπεύθυνος για την απόκτηση των δεδομένων από το model και την παρουσίαση τους στο view, αλλά σε αντίθεση με την κλασσική αρχιτεκτονική MVC είναι υπεύθυνος για τις αλλαγές που γίνονται στο view όταν ο χρήστης αλληλεπιδρά μαζί του.
	\end{itemize}
	
	Για να αποσαφηνιστεί πλήρως η λειτουργία του και να γίνει πιο εύκολα κατανοητό στο σχήμα \ref{fig:mvp_workflow} βλέπουμε ένα χαρακτηριστικό παράδειγμα.
	
	\begin{figure}[h]
	    \centering
	    \includegraphics[width=0.7\textwidth]{mvp_workflow.png}
	    \caption{Παράδειγμα MVP σε Android}
	    \label{fig:mvp_workflow}
	\end{figure}

	\subsection{Frameworks}
	
		Το framework, στον προγραμματισμό ηλεκτρονικών υπολογιστών, είναι μια αφηρημένη έννοια με την οποία κοινός κώδικας ο οποίος παρέχει γενικές λειτουργίες, μπορεί να επανεγγραφεί επιλεκτικά ή να εξειδικευτεί μέσω του κώδικα κάποιου χρήστη ώστε να χρησιμοποιηθεί για συγκεκριμένες λειτουργίες. Τα frameworks είναι μια ειδική περίπτωση βιβλιοθηκών λογισμικού που εμπεριέχουν επαναχρησιμοποιήσιμα κομμάτια κώδικα σε μια καλά καθορισμένη διεπαφή προγραμματισμού εφαρμογών (API), που όμως περιέχουν κάποια βασικά χαρακτηριστικά γνωρίσματα που τα διαφοροποιούν από τις κανονικές βιβλιοθήκες. Κάποια από τα γνωρίσματα αυτά είναι: αντιστροφή του ελέγχου (η ροή του προγράμματος υπαγορεύεται από το framework και όχι από αυτόν που το χρησιμοποιεί), προκαθορισμένη συμπεριφορά, επεκτασιμότητα (μπορεί να επεκταθεί από το χρήστη συνήθως με επανεγγραφή  ή εξειδίκευση του κώδικα ώστε να εκτελούνται συγκεκριμένες λειτουργίες. \cite{Framework}

	
		\subsubsection{Express Framework}

		Το Express.js είναι ένα framework του Node.js, σχεδιασμένο για να χτίζει μίας σελίδας, πολλών σελίδων, και hybrid web εφαρμογές. Είναι το προκαθορισμένο framework που χρησιμοποιείται πάνω στο node.js.  Ο δημιουργός του TJ Holowaychuck το περιέγραψε ως ένα "Sinatra inspired server", πράγμα που σημαίνει ότι είναι σχετικά ελάχιστο, με πολλά χαρακτηριστικά διαθέσιμα ως πρόσθετα. Το express είναι το backend της στοίβας MEAN.
		
		\subsubsection{Angular}
		Το AngularJS είναι ένα από τα πιο δημοφιλή ανοιχτού κώδικα JavaScript framework που χρησιμοποιείται για την κατασκευή και τη δομή σύγχρονων διαδικτυακών εφαρμογών, κυρίως εφαρμογών μίας σελίδας. Αναπτύχθηκε από την Google και από μία κοινότητα προγραμματιστών για την αντιμετώπιση των πολλών προκλήσεων που ανέκυψαν κατά την ανάπτυξη των εφαρμογών μίας σελίδας.  Στόχος του είναι να απλοποιήσει τόσο την ανάπτυξη όσο και το testing αυτών των εφαρμογών, παρέχοντας ένα framework για client-side model-View-Controller (MVC) και το model-view-viewmodel (MVVM) αρχιτεκτονικές, μαζί με τα συστατικά που χρησιμοποιούνται συνήθως στις εφαρμογές Διαδικτύου.
	   \\*
			    \textbf{Αρχιτεκτονική}
	 \\*
		Κατά την εκκίνηση της εφαρμογής, το πρόγραμμα περιήγησης φορτώνει το HTML αρχείο και το παρσάρει στο DOM αρχείο, και μαζί με αυτό στο angular.js script. Μόλις το DOM αρχείο έχει φορτωθεί, το AngularJS αναζητά την εντολή ng-app η οποία ορίζει τα όρια της εφαρμογής. Εάν ορίζεται κάποιο module μέσα στην οδηγία, τότε χρησιμοποιείται για να ρυθμιστεί το \$injector, που δημιουργεί τo \$compile service καθώς και το \$ rootScope. Τότε μεταγλωττίζεται το DOM και συνδέεται στο \$ rootScope μετά από το οποίο εκτελούνται οι υπόλοιπες εντολές (αν υπάρχουν). Στο σχήμα \ref{fig:angularjs} φαίνεται ο κύκλος εκκίνησης του AngularJS.
	  \begin{figure}[h]
	    \centering
	    \includegraphics[width=0.7\textwidth]{angularjs.png}
	    \caption{AngularJS κύκλος εκκίνησης.}
	    \label{fig:angularjs}
	\end{figure}

	Κάποια από τα βασικά χαρακτηριστικά του AngularJS είναι:
	\begin{itemize}
	\item Two Way Data-binding. Αυτό πρακτικά σημαίνει ότι όταν αλλάζει το model αλλάζει και το view, και όταν αλλάζει το  view αλλάζει και το model.  Το Two Way Data-binding εξηγείται καλύτερα με ένα παράδειγμα: 
	
	
	\lstinputlisting[language=Html]{two_way_data_binding.html}

	Αν ο χρήστης αλλάζει κάποια τιμή του model μέσω εισόδου στο view, τότε το AngularJs αποθηκεύει την τιμή
αυτή σε μια μεταβλητή. Το στοιχείο h1 στην συνέχεια ενημερώνεται αυτόματα ώστε να αντικατοπτρίζονται οι αλλαγές στην τιμή της μεταβλητής του model. Το view ενημερώνεται ακόμα και αν η μεταβλητή αλλάξει χειροκίνητα με χρήση JavaScript.\cite{angular}

	\item  HTML Templates. Ενώ άλλα JavaScript frameworks χρησιμοποιούν ένα template system που βασίζεται στην HTML με ειδική σήμανση το οποίο μπορεί να είναι δυσκολεύει τους προγραμματιστές, το AngularJS δεν στηρίζεται σε εξωτερικές template engines αλλά χρησιμοποιεί ένα templating σύστημα, που είναι χτισμένη πάνω στην HTML με έξυπνη χρήση των ng- χαρακτηριστικών. Το πρόγραμμα περιήγησης διαβάζει την HTML και "ψάχνει"  ng - directives τα οποία, όταν εκτελεστεί, δεσμεύουν το view στο model.

    \item Dependency Injection. Το dependency injection ενθαρρύνεται πολύ από το AngularJS και είναι ένα από τα συστατικά του, που βελτιώνει σε μεγάλο βαθμό το testability του. Το dependency injection είναι ένα design pattern λογισμικού στο οποίο σε ένα αντικείμενο ορίζονται οι εξαρτήσεις του και δεν τις δημιουργεί μόνο του. Πρόκειται ουσιαστικά για την αφαίρεση των hard-coded εξαρτήσεων κάνοντας δυνατό να αλλάξουν οι εξαρτήσεις όποτε χρειαστεί.
      Ένα παράδειγμα είναι όταν έχεις μία συνάρτηση η οποία παίρνει μία παράμετρο, όπως το ακόλουθο παράδειγμα:
    	
	\lstinputlisting[language=JavaScript]{angular_dependency_injection.js}

Το παραπάνω παράδειγμα είναι εύκολο να το τεστάρουμε, επειδή το στοιχείο που εισάγεται στην συνάρτηση μπορεί να είναι είτε ένα πραγματικό αντικείμενο ή ένα αντικείμενο για τους σκοπούς των τεστ. Ο διαχωρισμός σε μικρότερα πιο εύκολα ελέγξιμα συστατικά μπορεί να βοηθήσει πολύ στην ανίχνευση των σφαλμάτων.

    \item Deep Linking. Σε μια εφαρμογή μίας σελίδας είναι σημαντικό να διατηρείται η κατάσταση της εφαρμογής στο url, έτσι ώστε οι χρήστες είναι σε θέση να βάλουν κάποιο σελιδοδείκτη ή να μοιραστούν κάποιον σύνδεσμο σε διάφορες καταστάσεις της εφαρμογής.Το AngularJS χρησιμοποιεί το HTML5 history API  μαζί με ένα (\#!) εφεδρικό για παλαιότερα προγράμματα περιήγησης. Η λειτουργία αυτή είναι εξαιρετικά ισχυρή στη σύγχρονη εποχή των κοινωνικών δικτύων και του sharing.

	\item Directives. Τα directives είναι κάτι μοναδικό στο AngularJS και επιτρέπουν την επέκταση της λειτουργικότητας της HTML. Ενώ το AngularJS έχει μια συλλογή από προκαθορισμένα directives, μπορεί να επεκταθεί με συγκεκριμένες λειτουργικότητες σε σημείο που να επιτρέπει στον χρήστη να δημιουργήσει το δικό του DSL (Domain specific language). Επίσης επιτρέπει να δημιουργηθούν προσαρμοσμένα στοιχεία DOM, καθώς και χαρακτηριστικά και κλάσεις στα οποία μπορούν να προστεθούν λειτουργικότητες. Το επόμενο παράδειγμα από το επίσημο documentation του AngularJS δείχνει ότι για παράδειγμα γίνεται να επιτραπούν data-bindings για ένα html στοιχείο, αν υπάρχει ένα συγκεκριμένο χαρακτηριστικό. \cite{angular-directives}
\\*
\textbf{HTML:}

	\lstinputlisting[language=Html]{angular_directives.html}
 
\textbf{JavaScript:}

	\lstinputlisting[language=JavaScript]{angular_directives.js}


	\item \$http service. Το AngularJS είναι εξαιρετικά ευέλικτο στον τρόπο που επικοινωνεί με τα διαφορετικά backends. Αντί να βασίζεται αποκλειστικά στο REST interface, επικοινωνεί ελεύθερα μέσω των XMLHttpRequests του προγράμματος περιήγησης ή μέσω του JSONP. Παρόλο που το \$ http service προσθέτει http headers ως προεπιλογή στα αιτήματα του,  είναι εύκολα διαμορφώσιμα μέσω του \$httpProvider.defaults.headers.
	\end{itemize}




		\subsubsection{Bootstrap}
	Το bootstrap είναι ένα πάρα πολύ δυνατό front-end framework για την ταχύτερη και ευκολότερη ανάπτυξη ιστοσελίδων. Είναι μία συλλογή εργαλείων ανοιχτού κώδικα και περιλαμβάνει HTML και βασισμένα στο CSS πρότυπα σχεδιασμού για κοινά στοιχεία διεπαφής χρήστη, όπως τυπογραφίας, φόρμες, κουμπιά, πίνακες, dropdowns, alerts, μπάρες, και πολλά άλλα, καθώς και προαιρετικές επεκτάσεις JavaScript.Το bootstrap είναι συμβατό με όλες τις τελευταίες εκδόσεις των προγραμμάτων πλοήγησης και υποστηρίζει "responsive web design". Αυτό σημαίνει πως η διάταξη των ιστοσελίδων προσαρμόζεται δυναμικά, λαμβάνοντας υπόψιν τα χαρακτηριστικά της συσκευής που το χρησιμοποιεί (σταθερός υπολογιστής, κινητό, τάμπλετ κλπ.). 
	
	Το bootstrap είναι modular και αποτελείται από μικρότερα LESS stylesheets τα οποία υλοποιούν τα διάφορα στοιχεία του πακέτου εργαλείων. Ένα stylesheet, με όνομα bootstrap.less περιλαμβάνει τα επιμέρους stylesheets. Οι προγραμματιστές μπορούν να προσαρμόσουν και οι ίδιοι το bootstrap, καθώς τους δίνεται η δυνατότητα να επιλέξουν τα στοιχεία που θέλουν να χρησιμοποιήσουν. Αναπροσαρμογές γίνονται σε περιορισμένο όμως βαθμό, μέσω ενός κεντρικού stylesheet. Πιο ουσιαστικές αλλαγές πραγματοποιούνται με LESS δηλώσεις. Η χρήση των LESS stylesheets επιτρέπει τη χρήση μεταβλητών, συναρτήσεων καθώς και mixins. Επιπλέον, ο σχεδιαστής μπορεί να επιλέγει σε μια φόρμα τα επιθυμητά στοιχεία και να τα προσαρμόζει, σε τιμές διαφόρων εναλλακτικών λύσεων για τις ανάγκες του. Στη συνέχεια δημιουργείται ένα πακέτο που περιλαμβάνει ήδη το προ-χτισμένο CSS stylesheet.
	
	Το bootstrap παρέχει ένα σύνολο από stylesheets που παρέχουν βασικούς ορισμούς στυλ για όλα τα βασικά HTML στοιχεία με. Αυτά παρέχουν ενιαία, σύγχρονη εμφάνιση για πίνακες, μορφοποίηση κειμένου, καθώς και για τα στοιχεία μιας φόρμας. Επιπλέον των HTML στοιχείων, το bootstrap περιέχει και άλλα στοιχεία περιβάλλοντος τα οποία χρησιμοποιούνται συχνά. Αυτά περιλαμβάνουν κουμπιά με προηγμένα χαρακτηριστικά ( π.χ. ομαδοποίηση κουμπιών ή drop-down επιλογή, οριζόντιες και κάθετες καρτέλες, σελιδοποίηση, κ.λ.π. ), ετικέτες, προηγμένες τυπογραφικές δυνατότητες, εικονίδια, προειδοποιητικά μηνύματα και μια γραμμή προόδου. Τέλος, το bootstrap περιέχει πολλά συστατικά JavaScript σε μια μορφή jQuery plugin. Παρέχουν πρόσθετα στοιχεία διεπαφής χρήστη όπως παράθυρα διαλόγου, επεξηγήσεις, και καρουσέλ. Μπορούν επίσης να επεκτείνουν τη λειτουργικότητα ορισμένων υφιστάμενων στοιχείων της διασύνδεσης, όπως για παράδειγμα μια αυτόματη πλήρη λειτουργία για πεδία εισαγωγής. \cite{bootstrap}
	Τα πιο σημαντικά πλεονεκτήματα που προκύπτουν από την χρήση του bootstrap είναι:
	\begin{itemize}
	\item Ταχύτητα ανάπτυξης. Αδιαμφισβήτητα, ένα από τα σημαντικότερα πλεονεκτήματα της χρήσης του bootstrap είναι η ταχύτητα της ανάπτυξης. Το bootstrap παρέχει την δυνατότητα χρήσης έτοιμων μπλοκ κώδικα , έτσι ώστε να γίνεται η αρχή του front -end από εκεί και να μην χρειάζεται να γραφτεί κώδικας  από το μηδέν ("from scratch"). Το άνωθεν σε συνδυασμό με την συμβατότητα σε όλα τα προγράμματα περιήγησης που προσφέρει καθώς και τις λειτουργίες του CSS-LESS, εξοικονομεί πολύτιμο χρόνο στους προγραμματιστές.
	\item Ανταπόκριση. Λόγω του διαρκώς αυξανόμενου αριθμού των κινητών συσκευών υπάρχει η ανάγκη να έχουμε διαδραστικές ιστοσελίδες.Το bootstrap διευκολύνει πάρα πολύ την δημιουργία οθονών για κινητά χάρη στη ρευστή διάταξη πλέγματος που έχει που προσαρμόζεται δυναμικά στην κατάλληλη ανάλυση οθόνης και δεν χρειάζεται να κάνουμε επιπλέον δουλεία για να πετύχουμε την σωστή αποκρισιμότητα.
	\item Συνοχή. Το bootstrap βασίζεται σε αυτή την αρχή καθώς αρχικά αναπτύχθηκε από μερικούς υπαλλήλους του Twitter ως ένα framework για να υπάρχει συνέπεια μεταξύ των εσωτερικών εργαλείων που χρησιμοποιούνταν. 
Το bootstrap ουσιαστικά χτίστηκε ώστε να "ταιριάζει" τους σχεδιαστές με τους προγραμματιστές και να διασφαλίζει τη συνέπεια, ανεξάρτητα του ποιος εργάζεται στο έργο ή σε ποια πλατφόρμα θα τρέξει.
	\item Προσαρμόσιμα. Το bootstrap, μπορεί να προσαρμοστεί σύμφωνα με τις προδιαγραφές του κάθε έργου. Οι προγραμματιστές έχουν τη δυνατότητα να επιλέξουν τα χαρακτηριστικά που χρειάζονται και να πετάξουν τα υπόλοιπα.
	\end{itemize}

		
	\subsection{Express και Jade Template Engine}
	\subsection{Mockups}
	\subsection{Διασύνδεση με Social Networks}

\section{Mobile}
	\subsection{Mobile OS}
		Στην παρούσα ενότητα θα πραγματοποιήσουμε μια σύντομη σύγκριση μεταξύ των λειτουργικών συστημάτων iOS και Android. Περιορίζουμε την ανάλυση μόνο στα δύο προαναφερθείσα λειτουργικά γιατί αφενός έχουν τον μεγαλύτερο αριθμό ενεργών προγραμματιστών και αφετέρου τον μεγαλύτερο αριθμό χρηστών κάτι που τα καθιστά τα πλέον διαδεδομένα λειτουργικά στην αγορά των έξυπνων κινητών. Στον πίνακα \ref{tab:android_vs_ios} βλέπουμε μια στοιχειώδη σύγκριση μεταξύ των δύο λειτουργικών \cite{smartphoneMarketShare}\cite{androidPublish}\cite{applePublish}\cite{androidSource}\cite{androidWikipedia}\cite{iosWikipedia}.
		
	\begin{table}[H]
		\begin{center}
			\begin{tabular}{|l|c|c|}
			\hline
			\rowcolor{grayy}
			\textbf{Λειτουργικό} & \textbf{Android} & \textbf{iOS}
			\\ \hline
			Γλώσσα Προγραμματισμού & Java & Swift \\ \hline
			Περιβάλλον Ανάπτυξης & Android Studio & X-Code  \\ \hline
			Προσαρμοστικότητα & Μεγάλη & Ελάχιστη  \\ \hline
			Ανοιχτού Κώδικα & Ναι & Όχι  \\ \hline
			Μερίδιο Αγοράς & 82.8\% & 13.9\% \\ \hline
			Κόστος δημοσίευσης εφαρμογής & 25\$ εφάπαξ & 99\$/έτος  \\ \hline
			Έλεγχος Εφαρμογών & Όχι & Ναι \\ \hline
			\end{tabular}
			\caption{Σύγκριση βασικών στοιχείων Android \& iOS}
			\label{tab:android_vs_ios}
		\end{center}
	\end{table}
	Σε αυτό το σημείο πρέπει να αναφέρουμε ότι το iOS Developer Toolset λειτουργεί μόνο στα πλαίσια του λειτουργικού Mac OS X. Επομένως, η ανάπτυξη μιας εγγενής iOS εφαρμογής προϋποθέτει την ύπαρξη ενός Apple υπολογιστή. Δεδομένου του μεγαλύτερου μεριδίου αγοράς που διαθέτει το Android καθώς και τις ευκολίες που παρέχει προς τους προγραμματιστές, το επιλέξαμε για την πιλοτική έκδοση της εφαρμογής μας.
	\subsection{Android}
	Στην παρούσα ενότητα θα αναφερθούμε στα βασικά δομικά στοιχεία και στην αρχιτεκτονική του λειτουργικού συστήματος android.
		\subsubsection{Αρχιτεκτονική του Android}
		Το λειτουργικό σύστημα Android αποτελεί ουσιαστικά μια στοίβα πρωτοκόλλων για κινητά τερματικά που χωρίζεται σε πέντε κατηγορίες και τέσσερα επίπεδα όπως φαίνεται στο σχήμα \ref{fig:android_system_architecture}.
		
		\begin{figure}[h]
			\centering
			\includegraphics[width=0.9\textwidth]{android_system_architecture.jpg}
			\caption{Αρχιτεκτονική του λειτουργικού συστήματος Android}
			\label{fig:android_system_architecture}
		\end{figure}
		
		 Ξεκινώντας από το χαμηλότερο επίπεδο έχουμε \cite{collinsAndroid}\cite{annuzziAndroid}\cite{androidArchAnalysis}:
		 \begin{itemize}
		 	\item \textbf{Linux Kernel}: ο οποίος είναι ειδικά σχεδιασμένος για το λειτουργικό σύστημα Android και περιέχει τους οδηγούς για το υλικό (κάμερα, πληκτρολόγιο κτλπ). Επίσης ο πυρήνας είναι υπεύθυνος για τη διαχείριση της μνήμης και των διεργασιών, τις συνδέσεις δικτύου και κατ'επέκταση την διαχείριση των διεπαφών δικτύου που διαθέτει κάθε συσκευή. Επιπρόσθετα χειρίζεται τους πόρους και την ενεργεία κάτι το οποίο είναι πολύ σημαντικό δεδομένου της περιορισμένης παροχής ενέργειας των κινητών συσκευών. 
		 	\item \textbf{Libraries} Στο αμέσως επόμενο επίπεδο έχουμε τις βιβλιοθηκές του συστήματος, οι οποίες πολλές φορές αποκαλούνται και εγγενείς βιβλιοθήκες (Native Libraries). Οι εν λόγω βιβλιοθήκες έχουν αναπτυχθεί με χρήση των γλωσσών προγραμματισμού C και C++. Μερικές από τις πιο σημαντικές βιβλιοθήκες αυτού του επίπεδου είναι:
		 	\begin{itemize}
		 		\item WebKit, για την υποστήριξη των φυλλομετρητών
		 		\item Media Framework, για την αναπαραγωγή αρχείων ήχου και βίντεο
		 		\item Surface Manager, για την δημιουργία παραθύρων και ανανέωση της οθόνης
		 		\item SQLite, για την διαχείριση των βάσεων δεδομένων
		 		\item OpenGL, για την απόδοση γραφικών
		 		\item Text, για τον χειρισμό και εμφάνιση κειμένου
		 	\end{itemize}
		 	Σε αυτό το επίπεδο βρίσκεται επίσης και το κομμάτι της εκτέλεσης του Android (Android Runtime), το οποίο υποστηρίζει τη διαδικασία εγγραφής και εκτέλεσης των εφαρμογών. Αποτελείται από δυο βασικές συνιστώσες:
		 	\begin{itemize}
		 		\item Βασικές βιβλιοθήκες για τη διεπαφή των εφαρμογών Java με το περιβάλλον της συσκευής στην οποία εκτελούνται. Πιο αναλυτικά, οι εν λόγω βιβλιοθήκες είναι δυναμικές οι οποίες εισάγονται στην αρχή κάθε κλάσης.
		 		\item Την εικονική μηχανή Dalvik (Dalvik Virtual Machine) η οποία είναι υπεύθυνη για την δημιουργία των εκτελέσιμων αρχείων των εφαρμογών. Συγκεκριμένα ο πηγαίος κώδικας java μετατρέπεται σε μορφή ενδιάμεσου κώδικα (bytecode) και στην συνέχεια μεταφράζεται σε Dalvik bytecode που αποθηκεύεται σε αρχεία της μορφής .dex. Η διαδικασία της μετατροπής των αρχείων κλάσεων java σε μορφή .dex γίνεται από το εργαλείο dx, ενώ παράλληλα γίνεται και βελτιστοποίηση της πλεονάζουσας πληροφορίας με αποτέλεσμα τα αρχεία .dex να είναι μικρότερα σε μέγεθος από τα αντίστοιχα αρχεία κλάσεων. Τέλος το αρχείο .dex μαζί με τους πόρους της εφαρμογής μετατρέπονται σε αρχείο της μορφή .apk (Android Package). Το .apk αρχείο είναι αυτό που χρησιμοποιείται από τον χρήστη για την εγκατάσταση της εφαρμογής στην συσκευή του. Όταν ο χρήστης εκτελέσει την εφαρμογή, τότε η εφαρμογή αυτή θα εκτελεστεί στην εικονική μηχανή Dalvik αντί της JVM, καθώς το περιβάλλον εκτέλεσης των κινητών συσκευών διαθέτει περιορισμένους πόρους.
		 	\end{itemize}
		 	\item \textbf{Application Framework} Στο τρίτο επίπεδο βρίσκεται το πλαίσιο λογισμικού των εφαρμογών, το οποίο ουσιαστικά είναι ένα σύνολο υπηρεσιών οι οποίες δημιουργούν το περιβάλλον μέσα στο οποίο εκτελούνται οι εφαρμογές Android. Μερικές από τις πιο βασικές υπηρεσίες που περιλαμβάνονται στο επίπεδο του applicaiton framework είναι οι παρακάτω:
		 	\begin{itemize}
		 		\item Package Manager: Ο διαχειριστής πακέτων είναι το σύστημα μέσω του οποίου οι εφαρμογές μπορούν να βρουν πληροφορίες για τις υπόλοιπες εφαρμογές που είναι εγκατεστημένες στη συσκευή. Αποτελεί ουσιαστικά μια βάση δεδομένων όλων των εφαρμογών που είναι εγκατεστημένες στην συσκευή και επιτρέπει σε μια μια εφαρμογή να χρησιμοποιήσει μια άλλη και να μοιραστεί δεδομένα με αυτήν.
		 		\item Activity Manager: Ο διαχειριστής δραστηριοτήτων, διαχειρίζεται τον κύκλο ζωής των εφαρμογών και τη στοίβα δραστηριοτήτων (activity stack).
		 		\item Content Providers: Οι πάροχοι περιεχομένου επιτρέπουν στις εφαρμογές να δημοσιεύουν και να μοιράζονται δεδομένα με άλλες εφαρμογές. Αποτελούν ουσιαστικά βάσεις δεδομένων οι οποίες δίνουν την δυνατότητα στις εφαρμογές να αποθηκεύσουν και να μοιραστούν δομημένες πληροφορίες. Παράδειγμα τέτοιων πληροφοριών αποτελούν οι επαφές της συσκευής.
		 		\item View System: Περιλαμβάνει γραφικά στοιχεία τα οποία χρησιμοποιούνται για την δημιουργία της διεπαφής χρήστη. Παράδειγμα γραφικών στοιχείων που περιλαμβάνει είναι κουμπιά, κουτιά κειμένων, πλαίσια κτλπ.
		 		\item Resource Manager: Ο διαχειριστής πόρων είναι υπεύθυνος για την διαχείριση των πόρων που δεν αποτελούν κώδικα και κατ'επέκταση δεν μπορούν να μεταγλωττιστούν. Παράδειγμα τέτοιων στοιχείων αποτελούν τα γραφικά στοιχεία και οι συμβολοσειρές.
		 		\item Notification Manager: Ο διαχειριστής ειδοποιήσεων επιτρέπει στις εφαρμογές να εμφανίζουν ειδοποιήσεις προς το χρήστη. Οι ειδοποιήσεις αυτές εμφανίζονται στην μπάρα κατάστασης της συσκευής η οποία βρίσκεται στο πάνω μέρος της οθόνης και είναι σχεδόν πάντα ορατή στον χρήστη.
		 		\item Telephony Manager: Ο διαχειριστής τηλεφωνίας παρέχει πληροφορίες στην εφαρμογή σχετικά με τις υπηρεσίες τηλεφώνου που είναι διαθέσιμες στην συσκευή.
		 		\item Location Manager: Ο διαχειριστής τοποθεσίας παρέχει στις εφαρμογές πρόσβαση σε πληροφορίες σχετικά με την τοποθεσία και την κίνηση της συσκευής.
		 	\end{itemize}
		 	\item \textbf{Applications} Στο τέταρτο και τελευταίο επίπεδο της αρχιτεκτονικής του λειτουργικού συστήματος Android βρίσκονται οι εφαρμογές. Σε αυτό το επίπεδο περιλαμβάνονται τόσο οι προεγκατεστημένες εφαρμογές όσο και οι εφαρμογές τρίτων.
		 \end{itemize}
		 
		Υλοποίηση με Android, γιατί;
		Material design ? Αν δεν είναι API Level 21?
	\subsection{Mockups}
	\subsection{Διασύνδεση με Social Networks}
