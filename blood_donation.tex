\chapter{Εθελοντική αιμοδοσία}\label{ch:Βlood Donations}

\section{Εθελοντική αιμοδοσία}
	\subsection{Εισαγωγή}
	Το αίμα είναι το σπουδαιότερο βιολογικό υγρό του ανθρώπινου οργανισμού και αποτελεί ένα ανεκτίμητο προϊόν ζωής. Χωρίς της επαρκή ποσότητα αίματος, τα κύτταρα του ανθρώπινου σώματος δεν μπορούν να λάβουν το οξυγόνο και τα θρεπτικά συστατικά που τους είναι απαραίτητα για να επιβιώσουν \cite{aboutBlood}. Επιπλέον το σώμα δεν θα μπορούσε να αντιμετωπίσει βλαπτικούς παράγοντες, να αποβάλει τοξικά προϊόντα και να ρυθμίσει τις παραμέτρους του εσωτερικού του περιβάλλοντος \cite{circulatorySystem}. Αν και η τεράστια αξία και χρησιμότητα του αίματος είναι αδιαμφισβήτητη η ύπαρξη αρκετών εθελοντικών αιμοδοσιών για να καλυφθούν οι ανάγκες για αίμα αποτελεί μείζων πρόβλημα.
	
	Με τον όρο αιμοδοσία εννοούμε τη χορήγηση αίματος από υγιείς δότες σε άτομα στα
οποία η κατάσταση της υγείας τους απαιτεί μετάγγιση. Κατ' επέκταση με τον όρο αιμοδοσία εννοούμε την όλη οργάνωση που ασχολείται με τη λήψη, επεξεργασία, συντήρηση και διάθεση του αίματος και των Παραγώγων του \cite{bloodDonationDefinition}. Η αιμοδοσία καλείται εθελοντική, επειδή αφορά σε πράξη που εκτελεί κάποιος με τη θέλησή του και με μοναδικά κίνητρα αισθήματα αλληλεγγύης και αλτρουισμού \cite{1973}. 

Ως επιστημονικός τομέας, η αιμοδοσία αποτελεί ιδιαίτερο κλάδο της αιματολογίας με τεράστια ανάπτυξη τα τελευταία 20 χρόνια. Η ανάπτυξη της αιμοδοσίας ως εξειδικευμένου τομέα, καθώς και η αλματώδης ανάπτυξή της, οδήγησαν στην ανάγκη να πλαισιώνεται από ιατρικό, νοσηλευτικό και τεχνικό προσωπικό υψηλής στάθμης με εξειδίκευση στον τομέα της
αιμοδοσίας.  
	
	Η μετάγγιση αίματος γίνεται τακτικά σε εγχειρήσεις, τραυματίες, γαστρορραγίες και σε τοκετούς για την αναπλήρωση της απώλειας σημαντικής ποσότητας αίματος. Ακόμα, σε μερικές γενετικές διαταραχές, όπως αιμορροφιλία, η μεσογειακή αναιμία η οποία επηρεάζει το σχήμα των ερυθροκυττάρων και αυτό συνεπάγεται μείωση της λειτουργικότητας και συνακόλουθη καταστροφή τους. Ακόμη, σε ασθενείς με χρόνια νοσήματα όπως λευχαιμία, χρόνια νεφρική ανεπάρκεια, κακοήθεια. Τα άτομα που πάσχουν από αυτές τις διαταραχές χρειάζονται τακτικές μεταγγίσεις ασφαλούς αίματος και παραγώγων (πλάσμα, αιμοπετάλια) για την αναπλήρωση των παθολογικών συστατικών του αίματος τους.
	
	 Το αίμα που χρησιμοποιείται στις μεταγγίσεις πρέπει να προέρχεται από υγιή άτομα. Το αίμα δεν είναι μόνο ζωντανός ιστός, αλλά έχει επιπλέον την ιδιότητα να ανανεώνεται και τα υγιή άτομα διαθέτουν μηχανισμούς αύξησης της παραγωγής αίματος. Έτσι με την αιμοδοσία προσφέρεται εύκολα το δώρο της ζωής χωρίς το φόβο ότι η τακτική αιμοδοσία θα προκαλέσει εξασθένηση του οργανισμού και θα οδηγήσει σε αδυναμία η επιτάχυνση της γήρανσης.
	 
	Στόχος είναι οι εθελοντές, που πληρούν τα κριτήρια για αιμοδοσία να γίνονται τακτικοί αιμοδότες, δηλαδή να πραγματοποιούν δωρεά αίματος αρκετές φορές το χρόνο και να παραμένουν στον κατάλογο των ενεργών αιμοδοτών για πολλά χρόνια. Η διατήρηση ενός υψηλού επιπέδου ποιότητας παρεχόμενων υπηρεσιών στην υπηρεσία αιμοδοσίας συνίσταται στην προτεραιότητα ικανοποίησης των αναγκών και των προσδοκιών των εθελοντών αιμοδοτών. Ωστόσο είναι σημαντικό να επισημάνουμε ότι μια επένδυση στην προσέλκυση και την διατήρηση εθελοντών, δεν θα αποδώσει μόνο ασφαλή αποθέματα αίματος και προστασία της υγείας τόσο στον δότη όσο και στον λήπτη, αλλά και σημαντική εξοικονόμηση κόστους για την υπηρεσία μέσω της μείωσης του αριθμού μονάδων αίματος που πρέπει να απορριφθούν λόγω της ανεύρεσης θετικών δεικτών λοιμωδών νοσημάτων. 
	
	Επιπλέον η προσέλκυση και η διατήρηση των εθελοντών αιμοδοτών είναι μια δυναμική λειτουργία που σχεδιάζεται κάθε φορά ανάλογα με την μελέτη και ανάλυση των παραμέτρων της συγκεκριμένης κοινωνικής ομάδας που απευθυνόμαστε σε σχέση με την αξιολόγηση και εκτίμηση των αναγκών σε αίμα και την υπάρχουσα κατάσταση στο χώρο της.
	\subsection{Εθελοντές αιμοδότες}
		\subsubsection{Κριτήρια επιλογής αιμοδοτών} 
			Βασικός στόχος και υποχρέωση των υπηρεσιών μετάγγισης αίματος είναι να συλλέξουν αίμα από υγιείς αιμοδότες, ώστε αφενός να προφυλαχθεί η δική τους υγεία και αφετέρου να προστατευθεί ο αιμολήπτης ασθενής από τη μετάδοση ασθενειών ή φαρμακευτικών ουσιών που μπορεί δυνητικά να είναι βλαβερά για την υγεία του. Ο υποψήφιος αιμοδότης κατά τη λήψη του ιστορικού πρέπει να αναφέρει τυχόν συμπτώματα, ώστε να βοηθήσει το ιατρικό προσωπικό να κρίνει με ασφάλεια την πιθανότητα κινδύνου. Κάθε πρόβλημα υγείας που ενδεχομένως έχει ο υποψήφιος αιμοδότης, πρέπει να συζητείται με τον υπεύθυνο γιατρό της αιμοδοσίας, ο οποίος και κρίνει τελικά για τη καταλληλότητα της αιμοληψίας. Η διαδικασία θα πρέπει να είναι τέτοια έτσι ώστε και να εξασφαλίζεται ασφαλή αιμοδοσία και παροχή αίματος αλλά και να μην απορρίπτονται υγιείς αιμοδότες οι οποίοι θα μπορούσαν να συνεισφέρουν \cite{bloodDonorSelection}.
			
			Ο αιμοδότης πρέπει να είναι σε καλή υγεία και απαλλαγμένος από μεταδοτικές ασθένειες. Όμως κάθε άνθρωπος είναι επιρρεπής σε μικρο-αδιαθεσίες. Αυτές είναι πόνοι κάθε είδους, ακμή, πονόλαιμοι και δυσπεψία. Όλα αυτά δεν αποτελούν στοιχεία απόρριψης του αιμοδότη. Εάν ο δότης υποβάλλεται σε φαρμακευτική αγωγή ή έχει υποβληθεί στο άμεσο παρελθόν, πρέπει να σημειωθούν τα παρακάτω σχετικά με τα φάρμακα που παίρνει. Η ποσότητα του φαρμάκου, δηλαδή η πυκνότητα του φαρμάκου στον οργανισμό του δότη και η ταχύτητα απορρόφησης ή η αποβολή του. Το φάρμακο μπορεί να έχει δυσμενή επίπτωση στο δέκτη εφόσον η περιεκτικότητα του φαρμάκου στο δότη είναι αυξημένη. Εάν ο δέκτης είναι αλλεργικός σ' αυτό το φάρμακο ή εάν ο δέκτης είναι έγκυος γυναίκα μπορεί να προκληθεί μέχρι και τερατογένεση. Το φάρμακο μπορεί να διαταράξει το αίμα του δότη, π.χ. τη λειτουργικότητα των αιμοπεταλίων. Παρόλο ότι τα φαινόμενα αυτά έχουν αναγνωριστεί, η συχνότητά τους δεν μελετήθηκε ακόμη καλά. Εάν λοιπόν ο αιμοδότης λαμβάνει φάρμακα, δεν σημαίνει ότι αναγκαστικά δεν μπορεί να προσφέρει αίμα. Σε κάθε περίπτωση όμως είναι ορθό να ενημερώνεται ο γιατρός και το προσωπικό για τα φάρμακα που λαμβάνει και ανάλογα θα κριθεί εάν μπορεί να προσφέρει αίμα \cite{bloodDonorSelection}. 
			
		Υπάρχουν καταστάσεις και νοσήματα που αποκλείουν δια παντός την αιμοδοσία, όπως είναι το AIDS, οι ηπατίτιδες, η ελονοσία, η χρήση ενδοφλεβίων ναρκωτικών, οι κακοήθειες, η υπέρταση, ο σακχαρώδης διαβήτης ή σοβαρά χρόνια νοσήματα. Στις περισσότερες περιπτώσεις, όμως, ο αποκλεισμός είναι μόνο πρόσκαιρος.  Ο αποκλεισμός αυτός γίνεται για να μην επιβαρυνθεί η υγεία του αιμοδότη και για να διασφαλιστεί η ποιότητα του αίματος που θα μεταγγιστεί στο λήπτη. Μελέτες έχουν δείξει ότι ο προσωρινός αποκλεισμός από την αιμοδοσία έχει έντονο αρνητικό αντίκτυπο στην μελλοντική επιστροφή του αιμοδότη \cite{Custer2011}\cite{Custer2007}. Για αυτό είναι υψίστης σημασίας να δίνονται ξεκάθαρα διαστήματα αποκλεισμού και να ενθαρρύνονται οι εθελοντές να επιστρέψουν μετά το πέρας αυτού του διαστήματος. Στο σύστημα που προτείνουμε στη παρόν διπλωματική έχει γίνει μέριμνα για τέτοιες περιπτώσεις και προτείνετε ένα σύστημα κατάλληλων ειδοποιήσεων.
		
		Η καλή επικοινωνία μεταξύ του αιμοδότη και του προσωπικού της αιμοδοσίας διατηρεί τη σχέση εμπιστοσύνης και διασφαλίζει την καλή ποιότητα του αίματος που παρέχεται στους ασθενείς που χρειάζονται αίμα και τα παράγωγά του. Αίμα μπορούν να δώσουν όλοι οι υγιείς άντρες και γυναίκες ηλικίας 18 - 62 ετών, κάθε 3 - 4 μήνες. 
		\subsubsection{Κατηγορίες αιμοδοτών}
			NVD,LVD,RVD
	\subsection{Ανάγκες μονάδων αίματος}		
	\subsection{Εθελοντική αιμοδοσία στην Ελλάδα}
		Στην Ελλάδα το "Εθνικό Κέντρο Αιμοδοσίας" (Ε.ΚΕ.Α) το οποίο στεγάζεται στο Υπουργείο Υγείας και Πρόνοιας αποτελεί το κεντρικό όργανο για την οργάνωση των Υπηρεσιών Αιμοδοσίας. Οι Υπηρεσίες Αιμοδοσίας μπορούν να διαχωριστούν στις παρακάτω τρεις κατηγορίες: 1) τα Κέντρα Αιμοδοσίας, 2) τους Σταθμούς Αιμοδοσίας Α' Τάξης και 3) τους Σταθμούς Αιμοδοσίας Β' Τάξης. Τα κέντρα Αιμοδοσίας καλύπτουν τις ανάγκες μιας ευρύτερης γεωγραφικής περιοχής ή μεγάλων πληθυσμιακών ομάδων και εδρεύουν σε νοσοκομεία. Οι Σταθμοί Αιμοδοσίας Α' Τάξης είναι μικρότερες υπηρεσίες και καλύπτουν τις ανάγκες του νοσοκομείου στο οποίο εδρεύουν και άλλες τοπικές ανάγκες. Οι Σταθμοί Αιμοδοσίας Β' Τάξης καλύπτουν αποκλειστικά τις ανάγκες του νοσοκομείου που στεγάζονται.
		
		Το σύστημα αιμοδοσίας στην Ελλάδα είναι αποκεντρωμένο και αποτελείται από 101 υπηρεσίες αιμοδοσίας υπό την αιγίδα και εποπτεία του Υπουργείου Υγείας\cite{filloKivernisews}. Κάθε υπηρεσία αιμοδοσίας αποτελεί ένα ενσωματωμένο μέρος ενός δημόσιου νοσοκομείου και οι αρμοδιότητές της περιλαμβάνουν α) τη στρατολόγηση νέων αιμοδοτών, β) τη συλλογή και τον έλεγχο του αίματος και γ) και τη διακίνηση του αίματος και των παραγώγων του στις νοσοκομειακές κλινικές \cite{Marantidou2007}.
		
		Σύμφωνα με επίσημα δεδομένου του Υπουργείου Υγείας οι ανάγκες για αίμα το έτος 2014 στην Ελλάδα ήταν 750.000 μονάδες. Ένα σημαντικός παράγοντας ο οποίος αυξάνει σημαντικά τις ανάγκες σε αίμα στην χώρα μας και δικαιολογεί το παραπάνω νούμερο, είναι τα υψηλά ποσοστά μεσογειακής αναιμίας για τα οποία χρειάζονται 144.000 μονάδες αίματος ανά χρόνο. 
		
		Σύμφωνα με στοιχεία που έδωσε στη δημοσιότητα το Εθνικό Κέντρο Αιμοδοσίας (Ε.ΚΕ.Α) κατά την διάρκεια του έτους 2013 έγινε εθελοντική αιμοδοσία 584.088 μονάδων αίματος, εκ των οποίων 254.198 (43.52 \%) προήλθε από τους επονομαζόμενους δότες αντικατάστασης (Replacement Donors - RDs),οι οποίοι αιμοδοτούν προκειμένου να καλύψουν τις ανάγκες που προκύπτουν από συγγενείς ή φίλους. Ένα ποσοστό 54.85 \% δηλαδή 320.411 μονάδες προέρχονται από εθελοντές αιμοδότες (Volunteer Donors - VDs), οι οποίοι αιμοδοτούν με δική τους πρωτοβουλία καθαρά για ανθρωπιστικούς λόγους. Οι υπόλοιπες 9.479 μονάδες (0.016\%) προέρχονται από τις ένοπλες δυνάμεις. Η τελευταία αυτή κατηγορία αιμοδοτών έχει δυνατά κίνητρα να αιμοδοτήσει εθελοντικά, καθώς αποζημιώνονται με άδειες και αποχή από τα καθήκοντά τους. Σε αυτό το σημείο κρίνεται σκόπιμο να αναφερθεί ότι τα τελευταία τρία χρόνια έχει παρατηρηθεί αύξηση στις εθελοντικές αιμοδοσίες. Συγκεκριμένα το 2013 παρασχεθήκανε από εθελοντές αιμοδότες (VDs) 21.234 περισσότερες μονάδες αίματος σε σύγκριση με το 2012 \cite{EKEA}. Παρότι τα παραπάνω νούμερα είναι ενθαρρυντικά και παρότι διακρίνεται μια αυξητική τάση των εθελοντικών αιμοδοσιών, αξίζει να σημειωθεί ότι 24.000 μονάδες αίματος εισήχθησαν από την Ελβετία, προκειμένου να καλυφθούν οι εθνικές ανάγκες για αίμα \cite{Marantidou2007}.
		
		Βάση επίσημων στοιχείων, η Ελλάδα διαθέτει έναν ευρύ κατάλογο αιμοδοτών βάσει του οποίου 6 αιμοδότες αντιστοιχούν σε 100 πολίτες, γεγονός που την κατατάσσει τρίτη ανάμεσα στις χώρες- μέλη της Ευρωπαϊκής Ένωσης όσον αφορά τον αριθμό των ατόμων οι οποίοι έχουν δωρίσει αίμα έστω και μία φορά στη ζωή τους  (Eurobarometer, 2009).  Επίσης, η Ελλάδα έρχεται πρώτη όσον αφορά τους μη αιμοδότες, οι οποίοι όμως έχουν σκεφτεί να δώσουν αίμα (Eurobarometer, 2005). Παρόλα τα στοιχεία αυτά όμως, η Ελλάδα που είναι μία χώρα 11.000.000 κατοίκων, πολύ συχνά βρίσκεται στη δυσάρεστη αλλά αναπόφευκτη θέση να εισάγει αίμα από το εξωτερικό, καθώς ο ετήσιος αριθμός μονάδων αίματος δεν επαρκεί για να καλυφθούν οι ανάγκες της χώρας. Αυτό συμβαίνει λόγω των υψηλών ποσοστών μεσογειακής αναιμίας όπως αναφέραμε και παραπάνω αλλά και λόγω ότι οι ανάγκες αίματος κατά τη διάρκεια διάφορων χειρουργικών επεμβάσεων είναι μεγαλύτερη στην Ελλάδα από ότι σε άλλες χώρες της Κεντρικής και Βόρειας Ευρώπης, όπως προκύπτει από την έρευνα του ασφαλούς και
καλού αίματος \cite{Grindon1996} και προσπάθειες να ελαχιστοποιηθεί αυτή η αλόγιστη χρήση έχουν αποβεί προς το παρόν άκαρπες. 


\section{Πληροφοριακά συστήματα αιμοδοσίας}
	\subsection{Κατασταση στο εξωτερικο}
		Ανελυσε Καναδα και Ηνωμενο Βασιλειο που εχουν κανει καλη δουλεια
	\subsection{Αποθήκευση προσωπικών δεδομένων - security}
		(Ελλάδα και εξωτερικό)
	\subsection{Επεξεργασια Προσωπικών Δεδομένων}
	\subsection{Ασφάλεια ιατρικών δεδομένων και προστασία του απορρήτου του
ασθενούς}
		τι παιζει με την αρχη προστασιας δεδομενων ; (το αναφεραν στο εθνικο μητρωο αιμοδοτων - μαθε λεπτομερειες!)

\section{Κινδυνοι που προκύπτουν μέσα από αιμοδοσία (ασθένειες, ποσοστά)}
	\subsection{Haemovigilance}
Με τον όρο αιμοεπαγρύπνηση ορίζουμε ως ένα σύνολο οργανωμένων διαδικασιών επιτήρησης, που σχετίζονται με τα ανεπιθύμητα και μη αναμενόμενα συμβάντα και αντιδράσεις στους δότες και τους λήπτες των προϊόντων του αίματος και με την επιδημιολογική παρακολούθηση των αιμοδοτών. Η αιμοεπαγρύπνηση εστιάζει στις  επιπλοκές των δωρητών αίματος και στις ανεπιθύμητες αντιδράσεις των ληπτών και συμπεριλαμβάνει όλες τις επιπλοκές της “γραμμής παραγωγής αίματος”, τα ανεπιθύμητα συμβάματα, την αναλυτική και αναδρομική καταγραφή των γεγονότων και την προοπτική έγκαιρης προειδοποίησης με χρήση ενός συστήματος ταχείας έγερσης συναγερμών. Η αιμοεπαγρύπνηση είναι ένα ισχυρό εργαλείο το οποίο στοχεύει στην βελτίωση της ποιότητας των διαδικασιών μετάγγισης αίματος, δίνοντας προταιρεότητα στην ασφάλεια. Απώτερος στόχος της αιμοεπαγρύπνησης είναι η πρόληψη ανεπιθύμητων συμβαμάτων και αντιδράσεων. Εισαγάγει μεθόδους εντοπισμού σφαλμάτων, συμβαμάτων και αντιδράσεων και συμπεριλαμβάνει συστημάτα συναγερμού, συστήματα ιχνηλασιμότητας, συστημάτα ειδοποιήσεων και τους ελέγχους των πρακτικών.
		Οι ουσιαστικοί στόχοι της διαδικασίας της αιμοεπαγρύπνησης είναι οι εξής:
		\begin{itemize}
		\item Να προληφθεί η επανεμφάνιση ανεπιθύμητων συμβαμάτων και αντιδράσεων 
		\item Στη βελτίωση της πρακτικής μετάγγισης στα νοσοκομεία με τη λήψη προληπτικών ή διορθωτικών μέτρων όπου επιβάλλεται 
		\item Στην ανάπτυξη εθνικών κατευθυντήριων οδηγιών και νοσοκομειακών πρωτοκόλλων 
		\item Στη διαμόρφωση αποφάσεων σε εθνικό επίπεδο που αφορούν στην ασφάλεια των μεταγγίσεων 
		\item Στην εκπαίδευση των γιατρών που χρησιμοποιούν το αίμα, καθώς και των ασθενών που μεταγγίζονται 
		\end{itemize}

		Απαραίτητη προυπόθεση για την αποτελεσματική λειτουργία της αιμοεπαγρύπνησης είναι η 	ύπαρξη συστήματος ιχνηλασιμότητας. Ιχνηλασιμότητα ή ανιχνευσιμότητα (traceability) ορίζεται η ικανότητα πλήρους εντοπισμού καθεμίας μονάδας αίματος ή παραγώγων   		της από τον δότη μέχρι τον τελικό αποδέκτη. Επιτυγχάνεται  με ακριβείς και πλήρεις διαδικασίες αναγνώρισης κάθε δότη, κάθε συλλεγόμενης μονάδας αίματος, καθώς και όλων των παραγώγων του, τήρησης αρχείων και επισήμανσης καθώς και πλήρης διαδικασίας επαλήθευσης της παροχής και μετάγγισης αίματος.Με καταγραφή και μελέτη των απαραιτήτων  δεδομένων για την ιχνηλασιμότητα παράγουμε στατιστικά στοιχεία τα οποία αφορούν : στο  σύνολο των ασθενών που έχουν μεταγγισθεί , στις μονάδες ή στα παραγώγα αίματος που  έχουν χρησιμοποιηθεί, στους εθελοντές αιμοδότες που έχουν δώσει τις μονάδες αίμοτος και στα προϊόντα αίματος που μεταγγίσθηκαν. 
		Τα ανεπιθύμητα συμβάματα τα οποία μπορούν να προκύψουν χωρίζονται στις εξής βασικές 	κατηγορίες:
	
		\begin{itemize}
		\item	 Σοβαρή ανεπιθύμητη αντίδραση (Serious Adverse Reaction, SAR)
		 “μια άνευ προθέσεως αντίδραση του δότη ή του ασθενούς που σχετίζεται με τη συλλογή ή τη μετάγγιση αίματος ή παραγώγων του 			και η οποία είναι θανατηφόρα, απειλητική για τη ζωή, προκαλεί αναπηρία ή ανικανότητα ή έχει ως αποτέλεσμα ή παρατείνει τη 					νοσηλεία ή τη νοσηρότητα”, (Οδηγία 2002/98/ΕΚ)
 		\\item Σοβαρό ανεπιθύμητο συμβάν (Serious Adverse Event , SAE)
 		“κάθε ατυχές περιστατικό που σχετίζεται με τη συλλογή, τον έλεγχο, την επεξεργασία, την αποθήκευση και τη διανομή αίματος ή 				παραγώγου του, που θα μπορούσε να προκαλέσει το θάνατο, να απειλήσει τη ζωή, ή να προκαλέσει αναπηρία ή ανικανότητα ή να 				έχει ως αποτέλεσμα ή να παρατείνει τη νοσηλεία ή τη νοσηρότητα” , (Οδηγία 2002/98/ΕΚ)
 		\item Παρ’ ολίγον συμβάματα (“near miss” events)
 		 “σφάλματα που αν δεν ανιχνευθούν θα μπορούσαν να οδηγήσουν σε λανθασμένο προσδιορισμό ομάδας αίματος, αποτυχία
 		 ανίχνευσης ενός ερυθροκυτταρικού αντισώματος, ή σε διανομή, συλλογή ή χορήγηση εσφαλμένου ή ακατάλληλου προϊόντος
 		 αίματος ”, (Οδηγός, Συμβούλιο της Ευρώπης, Έκδοση 14) 
 		 \item Σφάλματα των μεταγγίσεων χωρίς συμβάματα (uneventful tranfusion errors)
 		 “η μετάγγιση οποιουδήποτε εσφαλμένου, ακατάλληλου ή μη ενδεικνυόμενου προϊόντος αίματος που δεν προκαλεί βλάβη στον 					λήπτη”(Οδηγός, Συμβούλιο της Ευρώπης, Έκδοση 14η)
 		 \end{itemize} 
		\subsubsection{Οδηγίες Ευρωπαϊκής και WHO}
		Όλες οι υπηρεσίες αιμοδοσίας και τα νοσοκομεία, υποχρεώνονται να κοινοποιούν στις αρμόδιες αρχές με κατάλληλη διαδικασία αναφοράς τις σοβαρές ανεπιθύμητες αντιδράσεις 	και συμβάματα τα οποία σχετίζονται με τη μετάγγιση αίματος. Η κοινοποίηση στις αρμόδιες αρχές των σοβαρών ανεπιθυμήτων αντιδράσεων και συμβάντων σχετικά με τη μετάγγιση αίματος απαιτείται από την Ευρωπαϊκή Νομοθεσία (Οδηγία 2005/61/ΕΚ) και κατ’ επέκταση από την Εθνική νομοθεσία κάθε κράτους μέλους της Ευρωπαϊκής Ένωσης με κατάλληλη διαδικασία αναφοράς από τις Υπηρεσίες Αιμοδοσίας και τα Νοσοκομεία, που μεταγγίζουν ασθενείς.  
		Όλες οι υπηρεσίες αιμοδοσίας και τα νοσοκομεία, υποχρεώνονται να κοινοποιούν στις 		αρμόδιες αρχές με κατάλληλη διαδικασία αναφοράς τις σοβαρές ανεπιθύμητες αντιδράσεις 	και συμβάματα τα οποία σχετίζονται με τη μετάγγιση αίματος.
Η κοινοποίηση στις αρμόδιες αρχές των σοβαρών ανεπιθυμήτων αντιδράσεων και συμβάντων σχετικά με τη μετάγγιση αίματος απαιτείται από την Ευρωπαϊκή Νομοθεσία (Οδηγία 2005/61/ΕΚ) και κατ’ επέκταση από την Εθνική νομοθεσία κάθε κράτους μέλους της Ευρωπαϊκής Ένωσης με κατάλληλη διαδικασία αναφοράς από τις Υπηρεσίες Αιμοδοσίας και τα Νοσοκομεία, που μεταγγίζουν ασθενείς.
 		 

	\subsection{Haemovigilance στο εξωτερικό}
	\subsection{Υπόσυστηματα Haemovigilance Ελλάδα}
